% ratio - 1610, 149, 54, 43, 32. Padrão: 43
% tamanho da fonte - 8pt, 9pt, 10pt, 11pt(padrão), 12pt, 14pt, 17pt, 20pt
\documentclass[aspectratio=149]{beamer}

% Tema do slide
\usetheme{Copenhagen}

% Preambulo
\usepackage[utf8]{inputenc}
\usepackage{amsmath}
\usepackage{amsfonts}
\usepackage{amssymb}
\usepackage{mathtools}
\usepackage{graphicx}
\usepackage[brazilian]{babel}
\usepackage{hyperref}
\usepackage{physics}
\setbeamercovered{transparent} % deixa os itens das listas transparentes(ver o comando de marcar o item ativo)
\beamertemplatenavigationsymbolsempty % tira os atalhos de navegação do slide(canto inferiro direito)
\renewcommand{\footnotesize}{\tiny} % configuração das notas de rodapé
\usepackage{lipsum} % lorem
%\usefonttheme{default} % fontes: structurebold, structuresmallcapsserif, serif e default
%\usecolortheme{beaver} % altera as cores do tema: default, beaver, beetle, seahorse, wolverine, lily, albatross, crane


% Páginas
%\setbeamerfont{page number in head/foot}{size=\large} % tamanho da fonte grande
\setbeamertemplate{page number in head/foot}[totalframenumber] % framenumber - só a pagina / totalframenumber - pagina atual/total de paginas

% Capa
\title{Criando Slides - Aula 1}
\author{Claudio Henrique}
\date{Maio 2023}

% Inicio do documento
\begin{document}

% Inserindo a capa
\maketitle

% Sumário
\begin{frame}
	\tableofcontents %[pausesections] % mesmo efeito dos itens na seção 2	
\end{frame}

% Conteudo
\section{Seção 1}
	% Cria um slide
	\begin{frame}{Seção 1 - Frame 1}
		% Lista de itens
		\begin{itemize}
			\item[a] Item 1 \footnote{Nota de rodapé}
			\item Item 2
			\item Item 3 \footnote[9]{Nota de rodapé}
		\end{itemize}
	\end{frame}
	
	\begin{frame}{Seção 1 - Frame 2}
		\lipsum[1-1]
	\end{frame}

\section{Seção 2}

	\begin{frame}{Seção 2 - Frame 1}
		\begin{enumerate} [<+- | alert@+>] % deixa o item ativo em vermelho
			\item Item
			\item Item
			\item Item
		\end{enumerate}
	\end{frame}
	
\section{Seção 3}
	\begin{frame}{Seção 3 - Frame 1}
		% Separando em colunas
		\begin{columns}
			\column{0.5\textwidth}
			\lipsum[1-1]
			
			\column{0.5\textwidth}
			\lipsum[1-1]
		\end{columns}
	\end{frame}
	
\section{Seção 4}
	% destaque de sentenças
	\begin{frame}{Seção 4 - Frame 1}
		\begin{block}{Titulo 1}
			Texto
		\end{block}
		
		\begin{alertblock}{Titulo 2}
			Texto
		\end{alertblock}
		
		\begin{exampleblock}{Titulo 3}
			Texto
		\end{exampleblock}
	\end{frame}
\end{document}




















