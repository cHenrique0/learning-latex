\documentclass{beamer}

% Tema do slide
\usetheme{Copenhagen}

% Preambulo
\usepackage[utf8]{inputenc}
\usepackage{amsmath}
\usepackage{amsfonts}
\usepackage{amssymb}
\usepackage{mathtools}
\usepackage{graphicx}
\usepackage[brazilian]{babel}
\usepackage{hyperref}
\usepackage{physics}
\setbeamercovered{transparent} % deixa os itens das listas transparentes(ver o comando de marcar o item ativo)
\beamertemplatenavigationsymbolsempty % tira os atalhos de navegação do slide(canto inferiro direito)

% Páginas
%\setbeamerfont{page number in head/foot}{size=\large} % tamanho da fonte grande
\setbeamertemplate{page number in head/foot}[totalframenumber] % framenumber - só a pagina / totalframenumber - pagina atual/total de paginas

% Capa
\title{Criando Slides - Aula 1}
\author{Claudio Henrique}
\date{Maio 2023}

% Inicio do documento
\begin{document}

% Inserindo a capa
\maketitle

% Sumário
\begin{frame}
	\tableofcontents %[pausesections] % mesmo efeito dos itens na seção 2
\end{frame}

% Conteudo
\section{Seção 1}
	% Cria um slide
	\begin{frame}{Seção 1 - Frame 1}
		% Lista de itens
		\begin{itemize}
			\item[a] Item 1
			\item Item 2
			\item Item 3
		\end{itemize}
	\end{frame}
	
	\begin{frame}{Seção 1 - Frame 2}
		Slide 2
	\end{frame}

\section{Seção 2}

	\begin{frame}{Seção 2 - Frame 1}
		\begin{enumerate} [<+- | alert@+>] % deixa o item ativo em vermelho
			\item Item
			\item Item
			\item Item
		\end{enumerate}
	\end{frame}

\end{document}




















