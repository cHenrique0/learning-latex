% aspectratio - 1610, 149, 54, 43, 32. Padrão: 43
% tamanho da fonte - 8pt, 9pt, 10pt, 11pt(padrão), 12pt, 14pt, 17pt, 20pt
\documentclass[]{beamer}

% Tema do slide
\usetheme{Copenhagen}

% Preambulo
\usepackage[utf8]{inputenc}
\usepackage{amsmath}
\usepackage{amsfonts}
\usepackage{amssymb}
\usepackage{mathtools}
\usepackage{graphicx}
\usepackage[brazilian]{babel}
\usepackage{hyperref}
\usepackage{physics}
\setbeamercovered{transparent} % deixa os itens das listas transparentes(ver o comando de marcar o item ativo)
\beamertemplatenavigationsymbolsempty % tira os atalhos de navegação do slide(canto inferiro direito)
\renewcommand{\footnotesize}{\tiny} % configuração das notas de rodapé
\usepackage{lipsum} % lorem
%\usefonttheme{default} % fontes: structurebold, structuresmallcapsserif, serif e default
%\usecolortheme{beaver} % altera as cores do tema: default, beaver, beetle, seahorse, wolverine, lily, albatross, crane

% Páginas
%\setbeamerfont{page number in head/foot}{size=\large} % tamanho da fonte grande
\setbeamertemplate{page number in head/foot}[totalframenumber] % framenumber - só a pagina / totalframenumber - pagina atual/total de paginas

% Capa
\title{Criando Slides - Aula 1}
\author[Claudio Henrique]{Claudio Henrique Velozo Alexandre}
\institute{Universidade Federal do Maranhão}
\date{\today}

% Inicio do documento
\begin{document}

% Inserindo a capa
% Forma 1
\frame{\maketitle} 
%\frame{\titlepage}

% Forma 2
%\begin{frame}
%	\maketitle
%\end{frame}

% Forma 3
%\maketitle

% Sumário
% Forma 1
%\begin{frame}{Sumário}
%	\tableofcontents %[pausesections] % mesmo efeito dos itens na seção 2	
%\end{frame}

% Forma 2
%\frame{\tableofcontents}

% Mostra o sumário a cada inicio de seção
\AtBeginSection[]{
	\begin{frame}{Sumário}
		\tableofcontents[currentsection]
	\end{frame}
}

% Conteudo
\section{Seção 1}
	% Cria um slide
	\begin{frame}{Seção 1 - Frame 1}
		% Lista de itens
		\begin{itemize}
			\item[a] Item 1 \footnote{Nota de rodapé}
			\item Item 2
			\item Item 3 \footnote[9]{Nota de rodapé}
		\end{itemize}
	\end{frame}
	
	\begin{frame}{Seção 1 - Frame 2}
		\lipsum[1-1]
	\end{frame}

\section{Seção 2}

	\begin{frame}{Seção 2 - Frame 1}
		\begin{enumerate} [<+- | alert@+>] % deixa o item ativo em vermelho
			\item Item
			\item Item
			\item Item
		\end{enumerate}
	\end{frame}
	
\section{Seção 3}
	\begin{frame}{Seção 3 - Frame 1}
		% Separando em colunas
		\begin{columns}
			\column{0.5\textwidth}
			\lipsum[1-1]
			
			\column{0.5\textwidth}
			\lipsum[1-1]
		\end{columns}
	\end{frame}
	
\section{Seção 4}
	% destaque de sentenças
	\begin{frame}{Seção 4 - Frame 1}
		\begin{block}{Titulo 1}
			Texto
		\end{block}
		
		\begin{alertblock}{Titulo 2}
			Texto
		\end{alertblock}
		
		\begin{exampleblock}{Titulo 3}
			Texto
		\end{exampleblock}
	\end{frame}

% Referencias Bibliográficas
\begin{frame}{Referencias Bibliográficas}
	\begin{thebibliography}{4} % 4 é a quantidade de referencias
		\beamertemplatebookbibitems % marcador de livro
		\bibitem{sobrenome} [1] Nome
		\newblock{\em Titulo}.
		\newblock{Editora $2001$}
		
		\setbeamertemplate{bibliography item}[triangle] % marcador triangulo
		\bibitem{sobrenome} [2] Nome
		\newblock{\em Titulo}.
		\newblock{Editora $2002$}
		
		\beamertemplateonlinebibitems % marcador de internet
		\bibitem{sobrenome} [3] Nome
		\newblock{\em Titulo}.
		\newblock{$<Link>$, acessado em...}
		
		\beamertemplatearticlebibitems % marcador de artigo
		\bibitem{sobrenome} [4] Nome
		\newblock{\em Titulo}.
		\newblock{Editora $2004$}
	\end{thebibliography}
\end{frame}

\end{document}




















