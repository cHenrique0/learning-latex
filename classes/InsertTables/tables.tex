\documentclass[a4paper, 12pt]{article}
\usepackage[top=2cm, bottom=2cm, left=2.5cm, right=2.5cm]{geometry}
\usepackage[utf8]{inputenc}
\usepackage{amsmath, amsfonts, amssymb}
\usepackage[portuguese]{babel}

\begin{document}

\begin{enumerate}
	\item Tabela \ref{tab-derivadas} apresenta derivadas básicas
	
	% l - à esquerda, r - à direita, c - ao centro. A quantidade de letras é a quantidade de colunas da tabela.
	% p{tam} - paragrafo{tamanho do paragrafo}
	% o tamanho pode ser dado em centimetros: 8cm, 10cm ou em porcentagem: 0.4\textwidth
	% & separa as colunas, \\ separa as linhas.
	% | - bordas verticais, \hline - bordas horizontais

	\begin{table}[!htb]
		\centering % centraliza a tabela
		\begin{tabular}{|c|c|} % cria a tabela
			\hline
			Função & Derivadas \\ \hline
			$f(x) = x^{n}$ & $f'(x) = nx^{n-1}$ \\ \hline
			$f(x) = \log_a x$ & $f'(x) = \dfrac{1}{x\ln a}$ \\ \hline
		\end{tabular}
		\caption{Tabela de derivadas.}
		\label{tab-derivadas}
	\end{table}
\end{enumerate}

\end{document}