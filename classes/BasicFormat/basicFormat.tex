% preambulo - contem informações gerais do arquivo
\documentclass[a4paper, 12pt]{article} % comando obrigatorio para criação de um documento latex
\usepackage[top=2cm, bottom=2cm, left=2.5cm, right=2.5cm]{geometry} % pacote de ajuda para configurações das margens
\usepackage{lipsum} % pacote para criar texto ficticios
\usepackage[utf8]{inputenc} % pacote para text enconding(acentos e etc)

% corpo do documento - onde fica o texto em si
\begin{document}
% \lipsum[2-4]
1. Inserindo fórmulas matematicas

Equação polinomial do 2º grau

Uma equação na forma $ax^2 + bx + c = d$, com $a \neq 0$ será chamada de equação do 2º grau.

A solução dessa equação é dada por $$x = \frac{-b \pm \sqrt{b^2 - 4ac}}{2a}$$

2. Formatando texto

\begin{center}
Alinhado ao centro
\end{center}

\begin{flushright}
Alinhado à direita
\end{flushright}

\begin{flushleft}
Alinhado à esquerda
\end{flushleft}

\textbf{Texto em negrito}

\textit{Texto em itálico}

\underline{Texto sublinhado}

\textbf{\textit{\underline{Os 3 juntos}}}

\end{document}